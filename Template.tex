\documentclass{article}

\usepackage{ctex}
\usepackage{listings}
\usepackage{color}
\usepackage{fontspec}
\usepackage{geometry}
\usepackage{amsmath}

\geometry{a4paper,scale=0.8}
\newfontfamily\Monaco{Monaco}

\definecolor{mygreen}{rgb}{0,0.6,0}
\definecolor{mygray}{rgb}{0.5,0.5,0.5}
\definecolor{mymauve}{rgb}{0.58,0,0.82}

\lstset{ 
  backgroundcolor=\color{white},     % choose the background color; you must add \usepackage{color} or \usepackage{xcolor}; should come as last argument
  basicstyle=\footnotesize\Monaco,   % the size of the fonts that are used for the code
  breakatwhitespace=false,         	 % sets if automatic breaks should only happen at whitespace
  breaklines=true,                 	 % sets automatic line breaking
  captionpos=top,              
  columns=flexible, texcl,      
  commentstyle=\color{mygreen},    	 % comment style
  deletekeywords={...},            	 % if you want to delete keywords from the given language
  escapeinside={\%*}{*)},          	 % if you want to add LaTeX within your code
  frame=single,	                   	 % adds a frame around the code
  keepspaces=true,                 	 % keeps spaces in text, useful for keeping indentation of code (possibly needs columns=flexible)
  keywordstyle=\color{blue},       	 % keyword style
  language=C++,                 	 % the language of the code
  morekeywords={*,...},            	 % if you want to add more keywords to the set
  numbers=left,                    	 % where to put the line-numbers; possible values are (none, left, right)
  numbersep=5pt,                   	 % how far the line-numbers are from the code
  numberstyle=\tiny\color{mygray}, 	 % the style that is used for the line-numbers
  rulecolor=\color{black},         	 % if not set, the frame-color may be changed on line-breaks within not-black text (e.g. comments (green here))
  showspaces=false,                	 % show spaces everywhere adding particular underscores; it overrides 'showstringspaces'
  showstringspaces=false,          	 % underline spaces within strings only
  showtabs=false,                  	 % show tabs within strings adding particular underscores
  stepnumber=1,                    	 % the step between two line-numbers. If it's 1, each line will be numbered
  stringstyle=\color{mymauve},     	 % string literal style
  tabsize=4,	                   	 % sets default tabsize to 4 spaces
  title=\lstname                   	 % show the filename of files included with \lstinputlisting; also try caption instead of title
}

\title{OI模板}
\author{panda\_2134}
\date{\today}

\begin{document}

	\makeatletter
	\lst@CCPutMacro
    	\lst@ProcessOther {"2D}{\lst@ttfamily{-{}}{-}}
    	\@empty\z@\@empty
	\makeatother

	\maketitle
	\newpage
	\tableofcontents
	\newpage

	\section{图的DFS树}
		\subsection{强连通分量}
		
		一有向图上每个点有非负权值,求一条路径,使得路径上点权值和最大。点和边都可以多次经过,但是权值只计入答案一次。
		
		Solution: 缩点后直接在DAG上DP.
		\lstinputlisting{GraphTheory/TarjanSCC.cpp}
		
		\subsection{桥和割点}
		\subsection{点双连通分量}
		\subsection{边双连通分量}

	\newpage
	\section{最短路}
		\subsection{Dijkstra}
			\subsubsection{Using \lstinline|std::priority\_queue|}
			\lstinputlisting{GraphTheory/Dijkstra-STL.cpp}
			\subsubsection{Using \lstinline|__gnu_pbds::priority_queue|}
			使用了扩展库 \lstinline|pb_ds| 中的配对堆,自带修改堆内元素操作,速度更快。仅在允许使用STL扩展时才使用。
			\lstinputlisting{GraphTheory/Dijkstra-pb_ds.cpp}
	\section{网络流}
		\subsection{最大流}
			\subsection{Dinic}
			\lstinputlisting{NetworkFlow/Dinic.cpp}

		\subsection{最小费用最大流}
			\subsubsection{zkw费用流}
			\lstinputlisting{NetworkFlow/zkw.cpp}
			\subsubsection{Primal Dual}
			\lstinputlisting{NetworkFlow/PrimalDual.cpp}
	\newpage

	\section{树}
		\subsection{倍增LCA}
		\lstinputlisting{Tree/DoublingLCA.cpp}
		\subsection{欧拉序列求LCA}
		\lstinputlisting{Tree/EulerTourLCA.cpp}
		\subsection{树链剖分}
		\lstinputlisting{Tree/HLD.cpp}
		\subsection{点分治}

	\newpage

	% Data Structure
	\section{单调数据结构}
		\subsection{单调队列(滑动窗口)}
		\lstinputlisting{Monotonic/SlidingWindow.cpp}
		\subsection{单调栈}

	\newpage

	\section{线段树}
		\subsection{Lazy-Tag} % [SDOI2017] 相关分析
			\paragraph{Solution:}
			暴力拆开式子后(或者根据《重难点手册》的结论),发现要维护区间的 $\sum x_i$, $\sum y_i$, $\sum x_i y_i$, $\sum x_i^2$, 同时要支持区间加和区间设置为 $S+i \text{ 和 } T_j$.
			在线段树上维护 $add_s, add_t, set_s, set_t$,然后推一推式子找出Lazy-tag更新主Tag的公式即可。
			几个坑点:
			\begin{enumerate}
				\item $add_s, add_t$ 标记在下推的时候,不能赋值,要累加!!!累加!!!累加!!!
				\item 只有 $set_s, set_t$ 用 $-1$ 来标记不存在, $add_s, add_t$ 必须用 $0$ 标记不存在!不然是给自己找麻烦,多出来各种特判!!!
			\end{enumerate}
			\lstinputlisting{SegTree/CorrelationAnalyse.cpp}
		\subsection{动态开点线段树}

		\subsection{可持久化线段树}

	\section{离线二维数点}
		\subsection{带修改}
			\subsubsection{静态:线段树+扫描线}
			\subsubsection{动态:CDQ分治}
	\section{在线二维数点}
		\subsubsection{动态:二维线段树}
		时间复杂度 $\text{插入} O(\lg^2 n) - \text{查询} O(\lg n)$
		空间复杂度 $O(n^2)$
		\subsubsection{动态:树状数组套动态开点线段树}
		\subsubsection{动态:树状数组套平衡树}
	\newpage

	\section{平衡树}
		\subsection{Treap}
		\subsection{Splay}

	\newpage

	\section{动态树}
		\subsection{Link-cut Tree}

	\newpage

	\section{字符串}
		\subsection{KMP字符串匹配}
		1-indexed
		% To Be Done
		\subsection{AC自动机}
		0-indexed
		\lstinputlisting{String/ACAutomaton.cpp}
		\subsection{后缀数组}
		0-indexed
		\lstinputlisting{String/SuffixArray.cpp}

	\newpage

	\section{Miscellaneous}
		\subsection{ST表}
		\subsection{Fenwick Tree}
		\subsection{左偏树}
	\newpage

	% Algorithm
	\section{悬线法}
		\subsection{Algorithm 1}
		\subsection{Algorithm 2}

	\newpage

	\section{莫队}
		\subsection{普通莫队}
		\subsection{带修改莫队}

	\newpage

	\section{分块相关}
		\subsection{分块}
		\subsection{区间众数}

	\newpage
	
	\section{数论}
		\subsection{线性求逆元}
			\paragraph{推导}
			令 $p = ki + r (0 \le r < i)$ \\
			则 $0 \equiv ki + r \: (\text{mod } p)$ \\
			$\Rightarrow ki \cdot i^{-1} r^{-1} + r \cdot i^{-1} r^{-1} \equiv 0$ \\
			$\Rightarrow i^{-1} \equiv -k \cdot r^{-1} \equiv p - \left\lfloor \frac{p}{i}\right\rfloor + p \text{ mod } i \: ( \text{mod } p )$
			
			\lstinputlisting{Math/inv.cpp}
		\subsection{线性筛}
			\subsubsection{求$\varphi(n)$}
			\subsubsection{求$\mu(n)$}
			\subsubsection{求$d(n)$}
		\subsection{扩展欧几里得定理}
		\subsection{中国剩余定理}
		\subsection{扩展欧拉定理}
		\subsection{Lucas定理}

\end{document}